迭代器是可以指向一定范围内元素的一种泛型数据(可以为任意类型),迭代器具有遍历、操作元素的能力。

可以将迭代器理解为指针的总称。

迭代器可以分为:输入、输出、前向、双向、随机读取、连续迭代器(contiguous C++17),通过 iterator\+\_\+traits 每个迭代器表征特性。

迭代器具有如下特性:复制构造(\+X a, X b(a)),赋值初始化(=),析构性,交换性,数值类特性(如 value\+\_\+type,difference\+\_\+type,reference......),反引用($\ast$it),自增。

迭代器是重要的,因为所有的\+S\+T\+L操作是基于迭代器的,迭代器允许以不显示声明(collction-\/agnostic)的方式遍历顺序元素。迭代器可以用于创建构造器(generator)。

为了有效利用迭代器,迭代器的特性必须定义,包含:difference\+\_\+type 表示两个迭代器差值的类型;value\+\_\+type 反引用迭代器得到这个类型,同时禁止对输出迭代器使用;pointer,reference,iterator\+\_\+category。

\subsection*{input iterators}

Input iterator add a few small requirements on top of a base iterator.\+It can read from the pointed-\/to element.\+Input iterator 仅仅适用于单通道算法(single pass algo),一旦自增,之前值的所有拷贝都可能无效,例子:steam iterator 从键盘输入得到字符,迭代器自增,原来的输入的字符已经不见了,不要去访问了。

\subsection*{output iterators}

output iterators can used in a sequential output operations,where each element pointed by the iterator is written a value only once, and then the iterator is incremented.

Algorithm requireing output iterators should be single-\/pass output algorithm.


\begin{DoxyItemize}
\item each iterator\textquotesingle{}s position is deferenced, once at most. lvalue derefenceable.\+Must be a class or pointer type.
\item equality and inequality may not be defined for output iterators(\+Not required or not guaranteed to be there)
\end{DoxyItemize}

\subsection*{forward iterators}

FI Can be used to acess the sequeue of elements in a range in the direction that goes from its begining towards its end.\+There is a key difference from input iterators that input iterators are only single-\/pass guarantedd,FI must be multi-\/pass guarantedd.\+If a FI satisfies the requirements for an output iterator,it is mutable forward iterator.(如果没有自增操作,则永远指向某个元素)

FI satisfies the input iterators, but don\textquotesingle{}t need to statisfy the output iterators.

\subsection*{bidirectional iterators}

与\+F\+I相比,可以执行自减操作。

\subsection*{Random-\/access iterators}

Random-\/access iterators can be used to acess elements at an arbitrary offset position(任意偏移位置),与指向的元素相关联,但是提供指针一样的功能(functionality)。功能上是最复杂的迭代器种类。

符合双向迭代器的特性,常数时间消耗的任意数量的偏移,

\subsection*{auxiliary iterator functions}

The iterator library offers some special functions that can be used universally, regardless of the type of iterator they are used on\+:next,prev,advance,distance

The library also provides special functions to access the iterators defined in containers\+: begin,end,rbegin,rend.\+They also are available with a c prefix for const.

\subsection*{iterator adapters(适配器)}

适配器接收(take in)迭代器,并且改变它的部分行为。例如,reverse\+\_\+iterator 接收双向迭代器使其逆向运行;move\+\_\+iterator 接收任意类型的input iterator反引用使其产生右值引用(如同 std\+::move 的使用)。

iterator header defines serveral special iterators for developers to use.\+例如:insert\+\_\+iterator, front\+\_\+insert\+\_\+iterator, back\+\_\+insert\+\_\+iterator, ostream\+\_\+iterator, istream\+\_\+iterator, istreambuf\+\_\+iterator, ostreambuf\+\_\+iterator. 